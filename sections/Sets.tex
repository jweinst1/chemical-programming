% Sets and theory math section

\chapter{Sets}

\par In order to understand the modeling behind chemical programming, you must first be acquainted with set theory and operations that can be performed on sets. The designs and models will utilize several types of sets, including partially ordered sets and multi-sets. This chapter is a fundamental guide toward understanding set theory relevant for chemical programming, by defining key relationships and properties. It defines sets, operations like unions, intersections, complements, and more. 

\par Sets serve as an efficient and simple representation of most concepts in chemical programming. Sets can be ordered or unordered. They can be distinct or indistinct. A set can be a subset or superset of another set. Set representation is a convenient method of expressing chemical processes in mathematical terms. Types of sets that are unique to chemical program models will also be defined and discussed in this chapter.

\section{What is a set?}

A set is a collection of one or more elements. It is usually denoted with curly braces, $\{\}$. A set's elements can be expressed as some series of $E_n$ elements.

$$
 S = \{E_n \dots E_k\}
$$

If an element, $E$ is contained in a set $S$, then,

$$
E \in S
$$

and if a set $S$ has only one member, element $E$, then,

\begin{align*}
E &\equiv S, \\
E &\equiv \{E\}
\end{align*}

Similarly, if an element is not in some set, this is written as $E \notin S$.
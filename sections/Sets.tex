% Sets and theory math section

\chapter{Sets}

\par In order to understand the modeling behind chemical programming, you must first be acquainted with set theory and operations that can be performed on sets. The designs and models will utilize several types of sets, including partially ordered sets and multi-sets. This chapter is a fundamental guide toward understanding set theory relevant for chemical programming, by defining key relationships and properties. It defines sets, operations like unions, intersections, complements, and more. 

\par Sets serve as an efficient and simple representation of most concepts in chemical programming. Sets can be ordered or unordered. They can be distinct or indistinct. A set can be a subset or superset of another set. Set representation is a convenient method of expressing chemical processes in mathematical terms. Types of sets that are unique to chemical program models will also be defined and discussed in this chapter.

\section{What is a set?}

A set is a collection of one or more elements. It is usually denoted with curly braces, $\{\}$. A set's elements can be expressed as some series of $E_n$ elements.

$$
 S = \{E_n \dots E_k\}
$$

If an element, $E$ is contained in a set $S$, then,

$$
E \in S
$$

and if a set $S$ has only one member, element $E$, then,

\begin{align*}
E &\equiv S, \\
E &\equiv \{E\}
\end{align*}

A set can be contained within some other set. If this is true, a set is a subset of another set. Let us have one set $s = \{a, b\}$, and another set, $S = \{a, b, c\}$.
% Expression of subset 
$$
\forall x  \{\exists x \in s \rightarrow \exists x \in S \}
$$

where x is an element, this can be described more conveniently as

$$
s \subseteq S
$$

A set that has no elements is called an empty set, denoted $\emptyset$. A set is an empty set when,

$$
\forall x \rightarrow \neg \exists x \in s
$$


If an element is not in some set, this is written as $E \notin S$. A distinct element, $E_i$ can have multiple set memberships. $E_i \in S_1 \land E_i \in S_2$ implies a membership in the first and second set. This is also called having membership in the intersection of both sets.

\subsection{Intersection}

An intersection is a set that is composed of the elements present in two or more sets. It is a separate set expressed as the elements present in both some set $S_a$ and $S_b$. It is denoted with a $\cap$, such that,

$$
S_a \cap S_b = \{x \mid x \in S_a \land x \in S_b\}
$$

For the intersection of an arbitrary number of sets, or for a collection, $C$, whose elements are entirely sets, we can write
% equation for arbitrary intersection
$$
\Bigl\{ x \in \bigcap\limits_{S \in C} S \Bigr\}
$$

Intersections are also a way in which two sets' equality can be determined. If the intersection of two sets is an empty set, then those two sets share no elements and are \textit{completely} unequal, $S_a \cap S_b = \emptyset$. Conversely, if the set of elements in one set OR the other set is equivalent to the intersection of those two sets, the sets are equal to each other. Elements that reside in either one set or another is called a \textit{union}.

\subsection{Unions}

A union is a set that represents the elements of two or more sets. It is denoted with a $\cup$ Specifically,

%union definition
$$
S_a \cup S_b = \{ x \mid x \in S_a \lor x \in S_b\}
$$ 

The union, in the case of two sets, $S_a$ and $S_b$ ,  is composed of three \textit{distinct} sets. The third set, aside from the elements in $S_a$ or $S_b$ is the intersection, $S_a \cap S_b$. The union, in this form, is expressed as $S_a \cup S_b = \{S_a, S_b, (S_a \cap S_b)\}$. The membership of $S_a$ and $S_b$ and the exclusion from the intersection is called the symmetric difference of sets. Symmetric difference is the opposite of intersections, it judges how \textit{unlike} two sets are.

\subsection{Equality and Difference}

Two sets are equal if and only if their intersection is equal to their union.

$$
S_a = S_b \iff S_a \cap S_b = S_a \cup S_b
$$

This is true since the union of two sets is the set of all members across both of them, if every member of either set is in both sets, there is no such member that exists in one set and not the other. This idea can also be used to define set inequality. 

A set $S_a$ is not equal to a set $S_b$ when at least one element is not present in both sets,

\begin{align}
S_a &\neq S_b \iff \exists x \in S_a, x \notin S_b \\
S_a &\neq S_b \iff \exists x \in S_a \cup S_b, x \notin S_a \cap S_b
\end{align}

The set of elements in some set $S_a$ \textit{or} $S_b$ but not both is the symmetric difference of the sets, denoted with a $\triangle$

$$
S_a \triangle S_b = \{ x \mid x \notin (S_a \cap S_b) \land x \in S_a \lor x \in S_b  \}
$$

To relate the idea of set equality and difference, the intersection of two sets and their symmetric difference are \textit{opposites}. A ratio can be established between the count of two sets as a degree of equality, $Q_e$.

$$
Q_e = \frac{S_a \cap S_b}{S_a \triangle S_b}
$$ 

If $Q_e$ is 1, there are no members in the symmetric difference, and every member of the union is also a member of the intersection. If $Q_e$ is 0, this means there are no members in the intersection, and every member of the union is a member of the symmetric difference. The varying degrees between 1 or 0 can determine how similar a set is to another.

\section{Multisets}

Previously, we discussed sets with distinct collections of elements, such that for every unique member, only one of that member is present in the set. A basic set cannot contain multiplicities of it's members, while a multiset can. All basic sets can be expressed as multisets where each member has a multiplicity of 1. If $S = \{E_1 \dots E_n\}$ is a basic set, then a multiset can be expressed as:

% multiset definition
$$
M = \Bigl\{E^{m(e_1)}_1, \dots E^{m(e_n)}_n\Bigr\}
$$

where $m(e)$ is the multiplicity of some element $e$ in the set. An example multiset can also be written more simply as $\{a^2, b^1\}$, which contains two of the element $a$ and one of the element $b$. There are many varieties of multisets, as even sets with the same types of elements can carry different multiplicities.  Multisets are particularly important in chemical programming as they mirror the representation of a chemical compound, such as water, $H_2O$. A molecule of water can be expressed as a multiset with two hydrogen atoms and one oxygen atom.

\subsection{Membership}

With basic sets, membership is defined as a simple boolean value, whether or not a set contains some element,
 % basic member sets
$$
x \in S
$$

In a multiset, membership is an integer value, where 0 indicates no membership, and 1 or more corresponds to the element's multiplicity.
% multiset membership
$$
x \in M \iff \exists x : m(x) \geq 1
$$

Conversely, not being in the multiset is defined as the multiplicity being zero, $ x \notin M \iff m(x) = 0$. The multiplicity of an element also determines how much of the set it represents. An element with the highest multiplicity in the set is the most common element. An element is the \textit{mode} of the multiset if,

$$
x = mode(M) \iff \forall e \in M \rightarrow m(e) < m(x)
$$

When all the members of a multiset have the same multiplicity, there is no mode. The judgement of membership based on a numeric value rather than a boolean value can also be expressed as a function, This function's domain is any possible member, and it's range is the multiplicity of the input element,

$$
M_f : e \rightarrow m
$$

An \textit{empty} multiset would always have a range of zero, such that,

$$
\emptyset = M_f : e \rightarrow 0
$$

However, for the purposes of chemical programming, multisets are best represented as collections rather than functions.

\subsection{Operations}

Multisets can undergo several operations that basic sets cannot. Basic set operations and their definitions such as a union or an intersection don't apply well to multisets, due to the absence of the idea of multiplicity. Such as with equality, two multisets could have the same members yet different multiplicities of each member.

The first operation of multisets is addition. The addition operation results in a new set where the multiplicities and variety of each member of both multisets are placed into one. The operation is binary and denoted with a $\uplus$.

$$
M_1 \uplus M_2 = \{ x, y \mid (x, m(x)) \in M_1, (y, m(y)) \in M_2 \}
$$

For example, if $M_1 = \{a, a, b\}$, and $M_2 = \{c, c, b\}$, the addition set is, $\{a, a, b, b, c, c\}$. Multisets can also be subtracted from one another. The subtraction of two multisets mirrors the symmetric difference of two basic sets. It is a multiset of the elements, that do not have equal multiplicities between some multiset $M_1$ and multiset $M_2$.

% multiset difference
$$
M_1 - M_2 = \{ (x, m(x)) \mid m_1(x) - m_2(x) \neq 0 \rightarrow x \in M_1, x \in M_2 \}
$$

Two multisets are equal to each other if and only if their difference set is the empty multiset.

% definition of multiset equality 
$$
M_1 = M_2 \iff M_1 - M_2 = \emptyset
$$

Similar to basic sets,  the size of the difference set indicates how alike or different two multisets are. The larger the difference set, the more elements that $M_1$ and $M_2$ \textit{do not} share or have the same number of. The smaller the difference set, the more equal multiplicities shared between $M_1$ and $M_2$.

\section{Nested Sets}

Sets can contain members that are also sets themselves. Such a set that is contained within another set can be treated as a unique, distinct value, as opposed to a collection. Normally, sets are not regarded singularly. The idea of a subset, such as $S_1 \subseteq S_2$ reads that all the elements in $S_1$ are also in $S_2$. Yet this does not handle the case of a set with all the same members as $S_1$ being contained in $S_2$. For basic set membership, the following would hold true:



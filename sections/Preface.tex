% Preface Section

\section{Preface}
% \addcontentsline{toc}{section}{Preface}
\paragraph{   } This book is intended to serve as a reference and guide toward programming in a chemical paradigm.  It serves three main purposes. The first, and most important, is addressing the methods and models in which chemical processes can be utilized for programming. Chemical programming is an extremely young paradigm, with much to explore and create. Parts of this text will teach concepts of how to program chemically in existing programming languages.
\par The second is to illustrate design patterns that can be implemented in software and programming systems. Chemical programming offers an array of algorithms that can simplify common tasks such as sorting, filtering, searching and much more. The hope is to promote the development of chemical programming as a prominent paradigm in modern software development.  The third, is to detail the creation of machines that can process entirely chemical programs. This includes the creation and design of a chemical virtual machine, and paths to create  chemical programming languages.
\par This book is \textit{not} a tutorial in how to design programs that deal with real chemical process or chemical engineering. Programming techniques for dealing with problems in chemical analysis is also not a concern of this book. Some principles of chemistry will be discussed, but not developing programs to deal with actual chemical compounds or data. The focus here is not framed around scientific computing.

\newpage
% Introduction Section

\section{Introduction}

\par Chemical Programming is a paradigm of programming that is modeled after the mechanics of chemical reactions. This style focuses on utilizing the features of chemical components, such as atoms, elements, compounds and reactions. The intention of this programming paradigm is to allow the functionality and mechanisms in chemical processes to be applied to computational values. Chemical computation permits highly customizable abstraction over traditional types like booleans, integers, and strings.
% Paragraph on useful versus not useful parts of chemistry in CS
\par Purely modeling chemical reactions is not the goal, rather defining and modeling the components of chemistry to make them useful in programming. Computer science and chemistry are well defined, yet very different fields. The fashions and organizations of chemical reactions provide unique interfaces for program design. However, moles, molecular orbitals, and energy levels are not fruitful or applicable to abstract program designs, or a programming paradigm.